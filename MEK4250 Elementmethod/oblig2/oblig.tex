\documentclass[a4paper,norsk]{article}
\usepackage{preamble}
\usepackage{tabu}

\begin{document}
\maketitle

\section{Exercise 1}
In these set of exercises we will study the Stokes problem defined as 
\begin{align*}
-\Delta u + \nabla p = f \hspace{2mm} \text{in} \hspace{2mm} \Omega \\
\nabla \cdot v = 0 \hspace{2mm} \text{in} \hspace{2mm} \partial\Omega \\
u = g \hspace{2mm} \text{in} \hspace{2mm} \partial\Omega_N \\
\frac{\partial u}{\partial x} - pn = h \hspace{2mm} \text{in} \hspace{2mm} \partial\Omega_N\\
\end{align*}

First off we will define the weak formulation for the stokes problem. Let u $\in H_{D,g}^1$ and $p \in L^2$. Then
the stokes problem can be defined as 
\begin{align*}
a(u, v) + b(p, v) = f(v) \hspace{2mm} v \in H_{D,0}^1 \\
b(q, u) = 0 \hspace{2mm} q \in L^2 
\end{align*}
Where a and b defines the bilinear form, and f defines the linear form as

\begin{align*}
a(u, v) = \int \nabla u : \nabla v \hspace{1mm} dx \\
b(p, v) = \int p \nabla \cdot v \hspace{1mm} dx \\
f(v) = \int f v \hspace{1mm} dx + \int_{\Omega_N} h v \hspace{1mm} ds
\end{align*}

Further we will define to properties which will be usefull for solving the exercises \newline
\textbf{Cauchy-Schwarts inequality} \\
Let V be a inner product space, then
\begin{align*}
 |\langle u \,, v \rangle| \hspace{1mm} \leq \hspace{1mm} ||u|| \cdot ||w|| \hspace{2mm} \forall \hspace{1mm} v,q \in V
\end{align*}
\newline
\textbf{Poincare's Inequality}
Let $v \in H_0^1(\Omega)$
\begin{align*}
||v||_{L^2(\Omega)} \leq C |v|_{H^1} (\Omega) 
\end{align*}

\newpage
\subsection*{Exercise 7.1}
In this section we are to prove the conditions (7.14-7.16) from the course lecturenotes. 
Starting off with the first 
\textbf{Condition 7.14}
\begin{align*}
a(u_h, v_h) \leq C_1 ||u_n||_{V_n} ||v_n||_{V_n} \hspace{2mm} \forall u_n, v_n \in V_n
\end{align*}
As for in all of these conditions we will assume that $V_h \in H_0^1$, and for later conditions that$Q_h \in L^2$.
\newline
First off we write out the term $a(u_h, v_h)$, and we observe we can use the Cauchy-Schwarts inequality since
V is an inner product space. 
\begin{align*}
a(u_h, v_h) = \int_\Omega \nabla u_h : \nabla v_h \hspace{1mm} dx  = \langle \nabla u_h \,, \nabla v_h \rangle \\
\langle \nabla u_h \,, \nabla v_h \rangle \leq |\langle \nabla u_h \,, \nabla v_h \rangle|_0 \leq ||\nabla u_h||\cdot ||\nabla v_h||
\end{align*}
Now since we have defined that $u_h, v_h \in V_h \in H_0^1$ we can use the Poincare inequality. First we observe that

\begin{align*}
||\nabla u_h ||^2_{L^2} \leq ||u_h||^2_{H^1} = ||u_h||^2_{L^2} + ||\nabla u_h||^2_{L^2} \leq C_1 |u_h|^2_{H_1} +  ||\nabla u_h||^2_{L^2} =
C_1 ||\nabla u_h||^2_{L2} +  ||\nabla u_h||^2_{L^2} \\
(C_1 + 1) ||\nabla u_h||^2_{L2} \leq D ||u_h||^2_{H^1} 
\end{align*}



\textbf{Condition 7.14}
\begin{align*}
b(u_h, q_h) \leq C_2 ||u_h||_{V_h} ||q_h||_{Q_h} \hspace{2mm}  V_h \in H_0^1, \hspace{2mm} Q_h \in L^2,
\end{align*}
By direct insertion we get 

\begin{align*}
b(u_h, q_h) = \int p \nabla \cdot v \hspace{1mm} dx = \langle p \,, \nabla \cdot u \rangle \leq | \langle p \,, \nabla \cdot u \rangle |_0
\end{align*}
Using the Cauchy-Schwarts inequality we can show that
\begin{align*}
| \langle q \,, \nabla \cdot u \rangle | \leq ||q||_0 \cdot ||\nabla \cdot u||_0
\end{align*}
Hence, it holds to show that
\begin{align*}
  ||q||_0 \cdot ||\nabla \cdot u||_0 \leq C_2 ||u_h||_{1} ||q_h||_{0} \\
  ||\nabla \cdot u||_0 \leq C_2 ||u_h||_{1} 
\end{align*}
We choose square the left side of the inequality and expand the norm, in hope of finding a term to
determine the upper bound of b. Applying the poincare inequality on line 2, we can determine that the bound must be
determined by some constant $C_2$
\begin{align*}
||\nabla \cdot u||_0^2 \leq ||u||_1^2 = ||u||_0^2 + ||\nabla u ||_0^2 \\
||u||_0^2 + ||\nabla u ||_0^2 \leq C_2^2 |u|_1 + ||\nabla u||_0^2 = (C_2^2 + 1)||\nabla u ||_0^2 = C_2^2||\nabla u ||_0^2
\end{align*}
Hence, to show the implied boundedness of b, it holds to show $||\nabla \cdot u||_0^2 \leq (C_2^2 + 1)||\nabla u ||_0^2$
For simplicity we write out the terms for the $R^2$ case, but the same proof can be showed for the general case $R^n$. 

\begin{align*}
 ||\nabla \cdot u||_0^2 = \int \big(\frac{\partial^2 u}{\partial x^2} + \frac{\partial^2 v}{\partial y^2}\big)^2 \hspace{1mm} dx \\
 ||\nabla u||_0^2 = \int \big(\frac{\partial^2 u}{\partial x^2}\big)^2 + \big(\frac{\partial^2 u}{\partial y^2} \big)^2 + 
 \big(\frac{\partial^2 v}{\partial x^2} \big)^2 + \big( \frac{\partial^2 v}{\partial y^2} \big)^2 \hspace{1mm} dx
\end{align*}
Remembering that since we are in a normed vector space V, the triangle inequality holds.
\begin{align*}
||x + y || \leq ||x|| + ||y|| \hspace{2mm} \forall x,y \in V
\end{align*}
Rearrangeing the terms we observe that 
\begin{align*}
 ||\nabla \cdot u||_0^2 = \int \big(\frac{\partial^2 u}{\partial x^2} + \frac{\partial^2 v}{\partial y^2}\big)^2 \hspace{1mm} dx \\
 ||\nabla u||_0^2 = \int \big(\frac{\partial^2 u}{\partial x^2}\big)^2 
 +  \big( \frac{\partial^2 v}{\partial y^2} \big)^2 \hspace{1mm} dx
 + \int \big(\frac{\partial^2 v}{\partial x^2} \big)^2 + 
  \big(\frac{\partial^2 u}{\partial y^2} \big)^2  \hspace{1mm} dx \\
  \sqrt{ \int \big(\frac{\partial^2 u}{\partial x^2} + \frac{\partial^2 v}{\partial y^2}\big)^2 \hspace{1mm} dx }\leq 
 \sqrt{ \int \big(\frac{\partial^2 u}{\partial x^2}\big)^2} 
 + \sqrt{ \big( \frac{\partial^2 v}{\partial y^2} \big)^2 \hspace{1mm} dx }
\end{align*}
Hence $||\nabla \cdot u||_0^2 \leq C_2^2 ||\nabla u ||_0^2$, and the inequality holds. \\
\textbf{Condition 7.16, Coersivity of a}
\begin{align*}
 a(u_h, v_h) \geq C_3 ||u_h||_{V_h}^2 \hspace{2mm} \forall u_h \in V_h
\end{align*}
Using the Cauchy-Schwarts inequality as in the other proofs, we get similar result

\begin{align*}
 a(u_h, v_h) = \int \nabla u : \nabla v \hspace{1mm} dx = \langle \nabla u_h \,, \nabla v_h \rangle \\
 |\langle \nabla u_h \,, \nabla v_h \rangle| \leq ||\nabla u|| \cdot ||\nabla v||
\end{align*}



\end{document}
