\documentclass[a4paper,norsk]{article}
\usepackage{preamble}

\begin{document}
\maketitle

\subsection*{What are the main categories of petrophysical logs. What do they read and what can we extract from them?}

\begin{description}
\item[Borehole log] Used to measure general characteristics of the hole such as diameter and inclination.
	\begin{description}
	\item[Dipmeter] Used to calculate the dip of strata.
	\item[Photoclinometer] Measures the deviation of the drillhole from true vertical
	\item[Calliper log] Measures the diameter of the hole, using a spring agains the surface.
	\end{description}
\item[SP log] SP or Self Potential is a logging system mainly ment two measure things. The first is to find permeable areas in the soil as well as the boundaries of these areas, while the second is to find the resistivity in fluid within this area $\rho_w$. We can visualize the measurement by thinking of two cups of water mixed with different consentration of salt. Setting a copper wire between these two fluids attached with a voltmeter, we can observe ions from the fluid with the highest concentration of salt to the one with less. Therefore it is required to have a conductive mudcake or fluids in the drillhole, to be able to take these measurements.
\item[Resistivity log] As its states is ment to measure the resistivity in a soil formation, in other words how resistant the soil is for electrons to move through it.
This can be further used to calculate the water saturation, fraction of pore space occupied by water $S_w$, within that formation. The amount of water stored within these pores effects the resistivity, as well as the salt content in the water.
The measurement of resistivity can be used to determine the quantity of hydrocarbons in a located reservoir, due to its relation to porosity as well as the amount of hydrocarbons occupying these pores.
\item[Radioactivity logs] Radioactive logs are mainly used to identify rock formations, aswell as in calculations to determine the porosity and density of these formations. These logs exploit the natural volumes of radioactive particles in the soil such as Thorium and Uranium. With this, rock formations can be identified by knowledge of expected radioactivity within different types of rock. For example shale rock usually have the highest radioactivity content and will often emitt the highest radioactive measurements. This type of logging equiment can be used even if the hole is cased in steel, while ordinary electrical logs will fall short. There are two types, one measuring the activity around the sonde by emitting radioactivity from the sonde itself, while the other measures radioactivity from the surounding formations.  
	\begin{description}
	\item[Passive Sonde logs]
	\item[Spectral $\gamma$ log] used to measure individual radioactive particles, allowing more precise measurements of rock formations and mainly shale formation. The Thorium compound is especially important due to the fact that it is more common to find in shales than other radioactive compounds. 
	\item[Emitting Sonde logs]
	\item[Formation density ($\gamma$-$\gamma$)] Used to locate certain rocks of interest by measuring returning $\gamma$ rays scattered from surrounding rock around the sonde emitting radiation.
	\item[Neutron log] Exploits the fact that emitted neutrons only slow down significantly when colliding with atoms of similar mass. Once absorbed, $\gamma$ rays will be emitted from the heavier atom, which the log measures. In this type of soil, that will often be hydrogen which we know are found in both water and oil in significant amounts. These fluids occupy the pores within rock formations, and as a result we can measure the porosity.
	\end{description}
\item[Sonic log] Sends seismic waves through the soil, and the time the waves uses through the soil are logged by a reciever. The time used by the waves are used to calculate porosity in the soil.
\end{description}
\subsection*{What is the top reservoir in this well?}
The top reservoir in this well is located at 2745m. Observing the crossing of neutron porosity and bulk density values is an important factor to locate the well. Firstly the decreasing neutron porosity values indicate areas of higher concentration og hydrogen atoms, which also means higher porosity. As we know, hydrogen atoms absorbes the emitted neutrons from the sonde, resulting in a reduced count of neutrons reaching the detector. While the decreasing bulk density tells us we have less mass occupying a unit volume, which also means higher porosity. So from the data, it seams this is the top of the reservoir.
 
\subsection*{What kind of hydrocarbons do we have in this reservoir? Gas or oil- explain The}
If we had mainly concentration of gas in this reservoir, we would expect a larger deviation between the neutron porosity and bulk density. This because measured gas gives a porosity lower than the true porosity, due to the fact of fewer hydrocarbons pr unit volume in gas. Hence I believe the majority of hydrocarbons is oil. 

\subsection*{What is the hydrocarbon water contact in this field? Explain why you have a transition interval - gradual reduction in HC stauration down towards the water zone}
We know that from neutron logs, water has a slightly lower neutron density than oil. Secondly we know that oil is lighter than water, so we can expect a slightly closer gap bewteen neutron porosity and bulk density, in the leap from oil to water. We also know that we should expect a decrease in resistivity due to the fact that water is a conductor, while oil have isolator properties. It seems from the data that we have a uneven mixture towards the end of the reservoir, but from what I see it seems that the hydrocarbon water contact is about 2782m. IKKE FERDIG

\subsection*{Explain the difference between free water level and oil water contact (OWC)}
\begin{itemize}
\item The \textbf{free water level} is defined as the surface between oil and water in the reservoir. 
\item \textbf{Oil water contact} is the transition zone from oil to water. This transition doesn't have to be a horizontal surface, but can be seen as a transition zone and can be tilted.
\end{itemize}

\subsection*{What is the net hydrocarbon portion (reservoir rock) relative to the total reservoir column in this well? This portion is called Net/Gross}
From the log it seems the total reservoir of oil is located about from 2745m to 2774m which yields 29meters, while the gross thickness of the reservoir is from 2744m to 2786m, or 42meters. This results in a Net/Gross of $\approx$ 0.69

\subsection*{What kind of reservoir do we have (what kind of rocks)?, and explain why, based on the petrophysical log readings}
Measuring the maximum and minimum GR values, and making an average of these values, we can roughly tell that the data is located around that average. From lectures we know that we can roughly say that values below this value is mostly sand formations, while values above goes over to shale types. Since most of the values in the reservoir is located slightly lower than this average, I conclude that the rock formation is mainly made of sand formations, with some shale compartments. \newline
Studying the sonic log we observe 3 spikes of smaller DT combined with decreased neutron porosity, which might indicate that we have calcite in these areas.

\subsection*{What are the main parameters that control the hydrocarbon porevolume in the reservoir}
Hydrocarbon porevolume (HCPV) can be estimated with the following formula \newline
HCPV = Gross Rock Volume * Net/Gross Ratio * Porosity * HC saturation, which then are the main parameters that effects the HCPV.

\subsection*{Can you explain the difference between resources and reserves in a hydrocarbon reservoir?}  
The difference of resources and reserves lies mainly in the time aspect of exploiting a hydrocarbon reservoir. We talk about resources when we have located an oil reservoir which may or may not be exploited commercially. Once an oilcompany has settled for an oil reserve which they want to exploit commercially now or in the future, we classify the oil as reserves. 

\subsection*{What is STOOIP and GIIP}
\begin{itemize}
\item \textbf{STOOIP} is an abbreviation for Stock Tank Original Oil In Place, meaning the volume of oil located in a reservoir before initiating production.
\item \textbf{GIIP} is an abbreviation for Gas Initially In Place, meaning the volume of gas located in a reservoir before initiating production.
\end{itemize}
 
\subsection*{Give some comments and reflections why recovery factors in reservoir may vary a lot}
We know recovery factors in an oil reservoir depend on factors such as permeability, viscosity of the oil and pressure gradients within the reservoir. \newline
\indent In any given reservoir we would expect that even if we have good porosity conditions, the given rock formations within can in some areas obstruct fluid flow. Hence this will reduce the permeability and make it harder to recover the oil. 
\newline \indent In order to recover the oil and gas, we are dependent of a pressure gradient to overcome the viscous and gravity forces in the reservoir, to recover the oil. The natural pressure gradient is a result from pressure forces from water, rock formations and gas surrounding the oil in the reservoir. Once we start producing, we can expect a drop in the pressure due to more free volume from the recovered oil. This will in time make recovering process slow and in the end stop, due to the lack of pressure forcing the oil up. The pressure gradient will be different for each reservoir, but it's clear that if we want to keep the production up we have to keep the pressure up. We can do this by pumping in gas or liquids such as water in to the reservoir, to maintain the pressure and maintaining production.
\newline \indent Oil viscosity is also a factor in the recovering rate, due to the fact that higher viscosity will make it harder and slower to recover the oil. From the oil water contact up to the gas seal on the top, one can imagine that we will find oil with slightly different properties in the reservoir. So during production we can expect to get in contact with several types of oil which will change the rate of recovery, and ofcourse lower viscosity will result in higher production rate.

\subsection*{Make a calculation of porosity and oil saturation at level 2750m, 2763m, 2782m and 2790 in this well, based on petrophysical logs linked to these simplified equation}

Reorganizing the presented simplified equation, we get the following relation for Porosity
\[ \phi = \frac{\rho_b - \rho_{ma}}{\rho_f - \rho_{ma}} \]
Here $\rho_b$ is the density log reading, $\rho_f $ the density of the saturating fluid, and $\rho_{ma}$ the density of the matrix material. Combining this result with Archie's equation, we are able to calculate the oil saturation. For the requested depths I get the following results.
\begin{align*}
\begin{matrix}
\hspace{15mm} Meter \hspace{21mm} 2750 \hspace{4mm}2763 \hspace{4mm} 2782 \hspace{4mm} 2790\\
\hspace{15mm} Porosity \hspace{19mm} 0.22 \hspace{5mm} 0.36 \hspace{5mm} 0.27 \hspace{5mm} 0.121 \\
\hspace{9mm} Saturation \hspace{15mm} 0.77 \hspace{5mm} 0.85 \hspace{5mm} 0.73 \hspace{5mm} 1
\end{matrix}
\end{align*}

The accuracy of these calculations can be discussed due to the human factor of selecting values. The log data values are very rough, so selecting values for the calculations might give some errors. Ofcourse the instruments themselves can effect the log data, due to disturbence during the logging. Since most of the available logs of interest are presented, I think the only way to be able to get more exact values is to extract a core sample from the rock formation.


\subsection*{Reference}
[Cambridge Musset, Khan (2000). Looking into the earth Subsurface Geophysics, Chapter 18: 285–305.
\newline http://www.spe.org/industry/increasing-hydrocarbon-recovery-factors.php

\end{document}